\PassOptionsToPackage{unicode=true}{hyperref} % options for packages loaded elsewhere
\PassOptionsToPackage{hyphens}{url}
%
\documentclass[]{article}
\usepackage{lmodern}
\usepackage{amssymb,amsmath}
\usepackage{ifxetex,ifluatex}
\usepackage{fixltx2e} % provides \textsubscript
\ifnum 0\ifxetex 1\fi\ifluatex 1\fi=0 % if pdftex
  \usepackage[T1]{fontenc}
  \usepackage[utf8]{inputenc}
  \usepackage{textcomp} % provides euro and other symbols
\else % if luatex or xelatex
  \usepackage{unicode-math}
  \defaultfontfeatures{Ligatures=TeX,Scale=MatchLowercase}
\fi
% use upquote if available, for straight quotes in verbatim environments
\IfFileExists{upquote.sty}{\usepackage{upquote}}{}
% use microtype if available
\IfFileExists{microtype.sty}{%
\usepackage[]{microtype}
\UseMicrotypeSet[protrusion]{basicmath} % disable protrusion for tt fonts
}{}
\IfFileExists{parskip.sty}{%
\usepackage{parskip}
}{% else
\setlength{\parindent}{0pt}
\setlength{\parskip}{6pt plus 2pt minus 1pt}
}
\usepackage{hyperref}
\hypersetup{
            pdftitle={Week 2 exercises},
            pdfauthor={Daniel Alonso},
            pdfborder={0 0 0},
            breaklinks=true}
\urlstyle{same}  % don't use monospace font for urls
\usepackage[margin=1in]{geometry}
\usepackage{color}
\usepackage{fancyvrb}
\newcommand{\VerbBar}{|}
\newcommand{\VERB}{\Verb[commandchars=\\\{\}]}
\DefineVerbatimEnvironment{Highlighting}{Verbatim}{commandchars=\\\{\}}
% Add ',fontsize=\small' for more characters per line
\usepackage{framed}
\definecolor{shadecolor}{RGB}{248,248,248}
\newenvironment{Shaded}{\begin{snugshade}}{\end{snugshade}}
\newcommand{\AlertTok}[1]{\textcolor[rgb]{0.94,0.16,0.16}{#1}}
\newcommand{\AnnotationTok}[1]{\textcolor[rgb]{0.56,0.35,0.01}{\textbf{\textit{#1}}}}
\newcommand{\AttributeTok}[1]{\textcolor[rgb]{0.77,0.63,0.00}{#1}}
\newcommand{\BaseNTok}[1]{\textcolor[rgb]{0.00,0.00,0.81}{#1}}
\newcommand{\BuiltInTok}[1]{#1}
\newcommand{\CharTok}[1]{\textcolor[rgb]{0.31,0.60,0.02}{#1}}
\newcommand{\CommentTok}[1]{\textcolor[rgb]{0.56,0.35,0.01}{\textit{#1}}}
\newcommand{\CommentVarTok}[1]{\textcolor[rgb]{0.56,0.35,0.01}{\textbf{\textit{#1}}}}
\newcommand{\ConstantTok}[1]{\textcolor[rgb]{0.00,0.00,0.00}{#1}}
\newcommand{\ControlFlowTok}[1]{\textcolor[rgb]{0.13,0.29,0.53}{\textbf{#1}}}
\newcommand{\DataTypeTok}[1]{\textcolor[rgb]{0.13,0.29,0.53}{#1}}
\newcommand{\DecValTok}[1]{\textcolor[rgb]{0.00,0.00,0.81}{#1}}
\newcommand{\DocumentationTok}[1]{\textcolor[rgb]{0.56,0.35,0.01}{\textbf{\textit{#1}}}}
\newcommand{\ErrorTok}[1]{\textcolor[rgb]{0.64,0.00,0.00}{\textbf{#1}}}
\newcommand{\ExtensionTok}[1]{#1}
\newcommand{\FloatTok}[1]{\textcolor[rgb]{0.00,0.00,0.81}{#1}}
\newcommand{\FunctionTok}[1]{\textcolor[rgb]{0.00,0.00,0.00}{#1}}
\newcommand{\ImportTok}[1]{#1}
\newcommand{\InformationTok}[1]{\textcolor[rgb]{0.56,0.35,0.01}{\textbf{\textit{#1}}}}
\newcommand{\KeywordTok}[1]{\textcolor[rgb]{0.13,0.29,0.53}{\textbf{#1}}}
\newcommand{\NormalTok}[1]{#1}
\newcommand{\OperatorTok}[1]{\textcolor[rgb]{0.81,0.36,0.00}{\textbf{#1}}}
\newcommand{\OtherTok}[1]{\textcolor[rgb]{0.56,0.35,0.01}{#1}}
\newcommand{\PreprocessorTok}[1]{\textcolor[rgb]{0.56,0.35,0.01}{\textit{#1}}}
\newcommand{\RegionMarkerTok}[1]{#1}
\newcommand{\SpecialCharTok}[1]{\textcolor[rgb]{0.00,0.00,0.00}{#1}}
\newcommand{\SpecialStringTok}[1]{\textcolor[rgb]{0.31,0.60,0.02}{#1}}
\newcommand{\StringTok}[1]{\textcolor[rgb]{0.31,0.60,0.02}{#1}}
\newcommand{\VariableTok}[1]{\textcolor[rgb]{0.00,0.00,0.00}{#1}}
\newcommand{\VerbatimStringTok}[1]{\textcolor[rgb]{0.31,0.60,0.02}{#1}}
\newcommand{\WarningTok}[1]{\textcolor[rgb]{0.56,0.35,0.01}{\textbf{\textit{#1}}}}
\usepackage{graphicx,grffile}
\makeatletter
\def\maxwidth{\ifdim\Gin@nat@width>\linewidth\linewidth\else\Gin@nat@width\fi}
\def\maxheight{\ifdim\Gin@nat@height>\textheight\textheight\else\Gin@nat@height\fi}
\makeatother
% Scale images if necessary, so that they will not overflow the page
% margins by default, and it is still possible to overwrite the defaults
% using explicit options in \includegraphics[width, height, ...]{}
\setkeys{Gin}{width=\maxwidth,height=\maxheight,keepaspectratio}
\setlength{\emergencystretch}{3em}  % prevent overfull lines
\providecommand{\tightlist}{%
  \setlength{\itemsep}{0pt}\setlength{\parskip}{0pt}}
\setcounter{secnumdepth}{0}
% Redefines (sub)paragraphs to behave more like sections
\ifx\paragraph\undefined\else
\let\oldparagraph\paragraph
\renewcommand{\paragraph}[1]{\oldparagraph{#1}\mbox{}}
\fi
\ifx\subparagraph\undefined\else
\let\oldsubparagraph\subparagraph
\renewcommand{\subparagraph}[1]{\oldsubparagraph{#1}\mbox{}}
\fi

% set default figure placement to htbp
\makeatletter
\def\fps@figure{htbp}
\makeatother


\title{Week 2 exercises}
\author{Daniel Alonso}
\date{November 19th, 2020}

\begin{document}
\maketitle

Importing libraries

\begin{Shaded}
\begin{Highlighting}[]
\KeywordTok{library}\NormalTok{(ggplot2)}
\KeywordTok{library}\NormalTok{(matlib)}
\end{Highlighting}
\end{Shaded}

We define the following function to calculate matrix powers (thanks
profe!):

\begin{Shaded}
\begin{Highlighting}[]
\NormalTok{matrixpower <-}\StringTok{ }\ControlFlowTok{function}\NormalTok{(M,k) \{}
  \CommentTok{# ARGUMENTS:}
  \CommentTok{# M: square matrix }
  \CommentTok{# k: exponent}
  \ControlFlowTok{if}\NormalTok{(}\KeywordTok{dim}\NormalTok{(M)[}\DecValTok{1}\NormalTok{]}\OperatorTok{!=}\KeywordTok{dim}\NormalTok{(M)[}\DecValTok{2}\NormalTok{]) }\KeywordTok{return}\NormalTok{(}\KeywordTok{print}\NormalTok{(}\StringTok{"Error: matrix M is not square"}\NormalTok{))}
  \ControlFlowTok{if}\NormalTok{ (k }\OperatorTok{==}\StringTok{ }\DecValTok{0}\NormalTok{) }\KeywordTok{return}\NormalTok{(}\KeywordTok{diag}\NormalTok{(}\KeywordTok{dim}\NormalTok{(M)[}\DecValTok{1}\NormalTok{]))  }\CommentTok{# if k=0 returns the identity matrix}
  \ControlFlowTok{if}\NormalTok{ (k }\OperatorTok{==}\StringTok{ }\DecValTok{1}\NormalTok{) }\KeywordTok{return}\NormalTok{(M)}
  \ControlFlowTok{if}\NormalTok{ (k }\OperatorTok{>}\StringTok{ }\DecValTok{1}\NormalTok{)  }\KeywordTok{return}\NormalTok{(M }\OperatorTok\StringTok{ }\KeywordTok{matrixpower}\NormalTok{(M, k}\DecValTok{-1}\NormalTok{)) }\CommentTok{# if k>1 recursively apply the function}
\NormalTok{\}}
\end{Highlighting}
\end{Shaded}

\hypertarget{exercise-1}{%
\subsection{Exercise 1}\label{exercise-1}}

\begin{Shaded}
\begin{Highlighting}[]
\NormalTok{P <-}\StringTok{ }\KeywordTok{c}\NormalTok{(}\FloatTok{0.1}\NormalTok{, }\FloatTok{0.4}\NormalTok{, }\FloatTok{0.5}\NormalTok{,}
       \FloatTok{0.4}\NormalTok{, }\FloatTok{0.6}\NormalTok{,   }\DecValTok{0}\NormalTok{,}
       \FloatTok{0.6}\NormalTok{,   }\DecValTok{0}\NormalTok{, }\FloatTok{0.4}\NormalTok{)}
\NormalTok{P <-}\StringTok{ }\KeywordTok{matrix}\NormalTok{(P,}\DataTypeTok{nrow=}\DecValTok{3}\NormalTok{,}\DataTypeTok{byrow=}\NormalTok{T)}
\end{Highlighting}
\end{Shaded}

First we solve \(\pi P = \pi\):

\(\pi (P-I) = (1,0, \dots, 0)\)

\((\pi_{1}, \pi_{2}, \pi_{3}) \begin{pmatrix} 1 & 0.4 & 0.5 \\ 1 & -0.4 & 0 \\ 1 & 0 & -0.6 \end{pmatrix} = (1,0,0)\)

Solving the system we get:

\(\pi = (\frac{6}{17}, \frac{15}{17}, \frac{10}{17})\)

Which is our stationary distribution.

\newpage

\hypertarget{exercise-2}{%
\subsection{Exercise 2}\label{exercise-2}}

\begin{Shaded}
\begin{Highlighting}[]
\NormalTok{stationary_dist <-}\StringTok{ }\ControlFlowTok{function}\NormalTok{(P) \{}
\NormalTok{    dim =}\StringTok{ }\KeywordTok{sqrt}\NormalTok{(}\KeywordTok{length}\NormalTok{(P))}
\NormalTok{    mat =}\StringTok{ }\KeywordTok{matrix}\NormalTok{(P,}\DataTypeTok{nrow=}\NormalTok{dim, }\DataTypeTok{byrow=}\NormalTok{T)}
\NormalTok{    A =}\StringTok{ }\NormalTok{mat }\OperatorTok{-}\StringTok{ }\KeywordTok{diag}\NormalTok{(dim)}
\NormalTok{    b =}\StringTok{ }\KeywordTok{c}\NormalTok{(}\DecValTok{1}\NormalTok{,}\KeywordTok{rep}\NormalTok{(}\DecValTok{0}\NormalTok{,dim}\DecValTok{-1}\NormalTok{))}
\NormalTok{    A[,}\DecValTok{1}\NormalTok{] <-}\StringTok{ }\KeywordTok{rep}\NormalTok{(}\DecValTok{1}\NormalTok{,dim)}
    \KeywordTok{print}\NormalTok{(}\StringTok{"The system is the following:"}\NormalTok{)}
    \KeywordTok{showEqn}\NormalTok{(A, b)}

    \KeywordTok{print}\NormalTok{(}\StringTok{"The solution is the following:"}\NormalTok{)}
\NormalTok{    pi <-}\StringTok{ }\NormalTok{matlib}\OperatorTok{::}\KeywordTok{Solve}\NormalTok{(A, b, }\DataTypeTok{fractions =} \OtherTok{TRUE}\NormalTok{)}
    \KeywordTok{return}\NormalTok{(pi)}
\NormalTok{\}}

\NormalTok{P <-}\StringTok{ }\KeywordTok{c}\NormalTok{(}\FloatTok{0.1}\NormalTok{, }\FloatTok{0.4}\NormalTok{, }\FloatTok{0.5}\NormalTok{,}
       \FloatTok{0.4}\NormalTok{, }\FloatTok{0.6}\NormalTok{,   }\DecValTok{0}\NormalTok{,}
       \FloatTok{0.6}\NormalTok{,   }\DecValTok{0}\NormalTok{, }\FloatTok{0.4}\NormalTok{)}

\KeywordTok{stationary_dist}\NormalTok{(P)}
\end{Highlighting}
\end{Shaded}

\begin{verbatim}
## [1] "The system is the following:"
## 1*x1 + 0.4*x2 + 0.5*x3  =  1 
## 1*x1 - 0.4*x2   + 0*x3  =  0 
## 1*x1   + 0*x2 - 0.6*x3  =  0 
## [1] "The solution is the following:"
## x1      =   6/17 
##   x2    =  15/17 
##     x3  =  10/17
\end{verbatim}

\begin{verbatim}
## [1] "x1      =   6/17" "  x2    =  15/17" "    x3  =  10/17"
\end{verbatim}

\hypertarget{exercise-3}{%
\subsection{Exercise 3}\label{exercise-3}}

\begin{Shaded}
\begin{Highlighting}[]
\NormalTok{P <-}\StringTok{ }\KeywordTok{c}\NormalTok{(  }\DecValTok{0}\NormalTok{, }\DecValTok{1}\OperatorTok{/}\DecValTok{3}\NormalTok{,   }\DecValTok{0}\NormalTok{, }\DecValTok{2}\OperatorTok{/}\DecValTok{3}\NormalTok{,}
       \DecValTok{2}\OperatorTok{/}\DecValTok{3}\NormalTok{,   }\DecValTok{0}\NormalTok{,   }\DecValTok{0}\NormalTok{, }\DecValTok{1}\OperatorTok{/}\DecValTok{3}\NormalTok{,}
         \DecValTok{0}\NormalTok{,   }\DecValTok{0}\NormalTok{,   }\DecValTok{1}\NormalTok{,   }\DecValTok{0}\NormalTok{,}
       \DecValTok{1}\OperatorTok{/}\DecValTok{3}\NormalTok{, }\DecValTok{2}\OperatorTok{/}\DecValTok{3}\NormalTok{,   }\DecValTok{0}\NormalTok{,   }\DecValTok{0}\NormalTok{ )}
\NormalTok{P <-}\StringTok{ }\KeywordTok{matrix}\NormalTok{(P,}\DataTypeTok{nrow=}\DecValTok{4}\NormalTok{,}\DataTypeTok{byrow=}\NormalTok{T)}
\end{Highlighting}
\end{Shaded}

\hypertarget{a---find-the-communication-classes-and-classify-the-states.}{%
\subsubsection{a - Find the communication classes and classify the
states.}\label{a---find-the-communication-classes-and-classify-the-states.}}

We have two communication classes

\hypertarget{b---find-the-set-of-stationary-distributions.}{%
\subsubsection{b - Find the set of stationary
distributions.}\label{b---find-the-set-of-stationary-distributions.}}

First we solve \(\pi P = \pi\):

\(\pi (P-I) = (1,0, \dots, 0)\)

\(\begin{pmatrix} \pi_{1} & \pi_{2} & \pi_{3} & \pi_{4} \end{pmatrix} \begin{pmatrix} 1 & \frac{1}{3} & 0 & \frac{2}{3} \\ 1 & -1 & 0 & \frac{1}{3} \\ 1 & 0 & 0 & 0 \\ 1 & \frac{2}{3} & 0 & -1 \end{pmatrix} = (1,0,0,0)\)

Solving the system we get:

\(\pi = (0, 0, \pi_{3}, 0)\)

Which is our stationary distribution, where \(\pi_{3}\) is a free
variable.

\hypertarget{c---how-is-the-long-run-behavior-of-the-chain-analyze-lim_ntoinfty-pn-with-r}{%
\subsubsection{\texorpdfstring{c - How is the long-run behavior of the
chain? Analyze \(\lim_{n\to\infty} P^{n}\) with
R}{c - How is the long-run behavior of the chain? Analyze \textbackslash{}lim\_\{n\textbackslash{}to\textbackslash{}infty\} P\^{}\{n\} with R}}\label{c---how-is-the-long-run-behavior-of-the-chain-analyze-lim_ntoinfty-pn-with-r}}

\begin{Shaded}
\begin{Highlighting}[]
\KeywordTok{matrixpower}\NormalTok{(P,}\DecValTok{50}\NormalTok{)}
\end{Highlighting}
\end{Shaded}

\begin{verbatim}
##           [,1]      [,2] [,3]      [,4]
## [1,] 0.3333333 0.3333333    0 0.3333333
## [2,] 0.3333333 0.3333333    0 0.3333333
## [3,] 0.0000000 0.0000000    1 0.0000000
## [4,] 0.3333333 0.3333333    0 0.3333333
\end{verbatim}

Given that there's 2 classes in this MC, our long run behavior shows the
same values for columns and rows in class 1 (nodes 1,2,4) but not for
class 2 (node 3).

\hypertarget{d---does-this-chain-have-a-limiting-distribution}{%
\subsubsection{d - Does this chain have a limiting
distribution?}\label{d---does-this-chain-have-a-limiting-distribution}}

No, this chain does not have a limiting distribution because
\(\lim_{n\to\infty} \alpha P^{n} \neq \lambda\)

\hypertarget{exercise-4}{%
\subsection{Exercise 4}\label{exercise-4}}

\(X =\) number of pairs of shoes at front door

\(Y =\) number of pairs of shoes at back door

\(\begin{pmatrix} \frac{3}{4} & \frac{1}{4} & 0 & 0 & 0 & 0 \\ \frac{1}{4} & \frac{1}{2} & \frac{1}{4} & 0 & 0 & 0 \\ 0 & \frac{1}{4} & \frac{1}{2} & \frac{1}{4} & 0 & 0 \\ 0 & 0 & \frac{1}{4} & \frac{1}{2} & \frac{1}{4} & 0 \\ 0 & 0 & 0 & \frac{1}{4} & \frac{1}{2} & \frac{1}{4} \\ 0 & 0 & 0 & 0 & \frac{1}{4} & \frac{3}{4} \end{pmatrix}\)

\end{document}
