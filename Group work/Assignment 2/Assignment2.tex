\PassOptionsToPackage{unicode=true}{hyperref} % options for packages loaded elsewhere
\PassOptionsToPackage{hyphens}{url}
%
\documentclass[]{article}
\usepackage{lmodern}
\usepackage{amssymb,amsmath}
\usepackage{ifxetex,ifluatex}
\usepackage{fixltx2e} % provides \textsubscript
\ifnum 0\ifxetex 1\fi\ifluatex 1\fi=0 % if pdftex
  \usepackage[T1]{fontenc}
  \usepackage[utf8]{inputenc}
  \usepackage{textcomp} % provides euro and other symbols
\else % if luatex or xelatex
  \usepackage{unicode-math}
  \defaultfontfeatures{Ligatures=TeX,Scale=MatchLowercase}
\fi
% use upquote if available, for straight quotes in verbatim environments
\IfFileExists{upquote.sty}{\usepackage{upquote}}{}
% use microtype if available
\IfFileExists{microtype.sty}{%
\usepackage[]{microtype}
\UseMicrotypeSet[protrusion]{basicmath} % disable protrusion for tt fonts
}{}
\IfFileExists{parskip.sty}{%
\usepackage{parskip}
}{% else
\setlength{\parindent}{0pt}
\setlength{\parskip}{6pt plus 2pt minus 1pt}
}
\usepackage{hyperref}
\hypersetup{
            pdftitle={Stochastic Processes: Assignment 1},
            pdfauthor={Group 1: Javier Esteban Aragoneses, Mauricio Marcos Fajgenbaun, Danyu Zhang, Daniel Alonso},
            pdfborder={0 0 0},
            breaklinks=true}
\urlstyle{same}  % don't use monospace font for urls
\usepackage[margin=1in]{geometry}
\usepackage{color}
\usepackage{fancyvrb}
\newcommand{\VerbBar}{|}
\newcommand{\VERB}{\Verb[commandchars=\\\{\}]}
\DefineVerbatimEnvironment{Highlighting}{Verbatim}{commandchars=\\\{\}}
% Add ',fontsize=\small' for more characters per line
\usepackage{framed}
\definecolor{shadecolor}{RGB}{248,248,248}
\newenvironment{Shaded}{\begin{snugshade}}{\end{snugshade}}
\newcommand{\AlertTok}[1]{\textcolor[rgb]{0.94,0.16,0.16}{#1}}
\newcommand{\AnnotationTok}[1]{\textcolor[rgb]{0.56,0.35,0.01}{\textbf{\textit{#1}}}}
\newcommand{\AttributeTok}[1]{\textcolor[rgb]{0.77,0.63,0.00}{#1}}
\newcommand{\BaseNTok}[1]{\textcolor[rgb]{0.00,0.00,0.81}{#1}}
\newcommand{\BuiltInTok}[1]{#1}
\newcommand{\CharTok}[1]{\textcolor[rgb]{0.31,0.60,0.02}{#1}}
\newcommand{\CommentTok}[1]{\textcolor[rgb]{0.56,0.35,0.01}{\textit{#1}}}
\newcommand{\CommentVarTok}[1]{\textcolor[rgb]{0.56,0.35,0.01}{\textbf{\textit{#1}}}}
\newcommand{\ConstantTok}[1]{\textcolor[rgb]{0.00,0.00,0.00}{#1}}
\newcommand{\ControlFlowTok}[1]{\textcolor[rgb]{0.13,0.29,0.53}{\textbf{#1}}}
\newcommand{\DataTypeTok}[1]{\textcolor[rgb]{0.13,0.29,0.53}{#1}}
\newcommand{\DecValTok}[1]{\textcolor[rgb]{0.00,0.00,0.81}{#1}}
\newcommand{\DocumentationTok}[1]{\textcolor[rgb]{0.56,0.35,0.01}{\textbf{\textit{#1}}}}
\newcommand{\ErrorTok}[1]{\textcolor[rgb]{0.64,0.00,0.00}{\textbf{#1}}}
\newcommand{\ExtensionTok}[1]{#1}
\newcommand{\FloatTok}[1]{\textcolor[rgb]{0.00,0.00,0.81}{#1}}
\newcommand{\FunctionTok}[1]{\textcolor[rgb]{0.00,0.00,0.00}{#1}}
\newcommand{\ImportTok}[1]{#1}
\newcommand{\InformationTok}[1]{\textcolor[rgb]{0.56,0.35,0.01}{\textbf{\textit{#1}}}}
\newcommand{\KeywordTok}[1]{\textcolor[rgb]{0.13,0.29,0.53}{\textbf{#1}}}
\newcommand{\NormalTok}[1]{#1}
\newcommand{\OperatorTok}[1]{\textcolor[rgb]{0.81,0.36,0.00}{\textbf{#1}}}
\newcommand{\OtherTok}[1]{\textcolor[rgb]{0.56,0.35,0.01}{#1}}
\newcommand{\PreprocessorTok}[1]{\textcolor[rgb]{0.56,0.35,0.01}{\textit{#1}}}
\newcommand{\RegionMarkerTok}[1]{#1}
\newcommand{\SpecialCharTok}[1]{\textcolor[rgb]{0.00,0.00,0.00}{#1}}
\newcommand{\SpecialStringTok}[1]{\textcolor[rgb]{0.31,0.60,0.02}{#1}}
\newcommand{\StringTok}[1]{\textcolor[rgb]{0.31,0.60,0.02}{#1}}
\newcommand{\VariableTok}[1]{\textcolor[rgb]{0.00,0.00,0.00}{#1}}
\newcommand{\VerbatimStringTok}[1]{\textcolor[rgb]{0.31,0.60,0.02}{#1}}
\newcommand{\WarningTok}[1]{\textcolor[rgb]{0.56,0.35,0.01}{\textbf{\textit{#1}}}}
\usepackage{graphicx,grffile}
\makeatletter
\def\maxwidth{\ifdim\Gin@nat@width>\linewidth\linewidth\else\Gin@nat@width\fi}
\def\maxheight{\ifdim\Gin@nat@height>\textheight\textheight\else\Gin@nat@height\fi}
\makeatother
% Scale images if necessary, so that they will not overflow the page
% margins by default, and it is still possible to overwrite the defaults
% using explicit options in \includegraphics[width, height, ...]{}
\setkeys{Gin}{width=\maxwidth,height=\maxheight,keepaspectratio}
\setlength{\emergencystretch}{3em}  % prevent overfull lines
\providecommand{\tightlist}{%
  \setlength{\itemsep}{0pt}\setlength{\parskip}{0pt}}
\setcounter{secnumdepth}{0}
% Redefines (sub)paragraphs to behave more like sections
\ifx\paragraph\undefined\else
\let\oldparagraph\paragraph
\renewcommand{\paragraph}[1]{\oldparagraph{#1}\mbox{}}
\fi
\ifx\subparagraph\undefined\else
\let\oldsubparagraph\subparagraph
\renewcommand{\subparagraph}[1]{\oldsubparagraph{#1}\mbox{}}
\fi

% set default figure placement to htbp
\makeatletter
\def\fps@figure{htbp}
\makeatother


\title{Stochastic Processes: Assignment 1}
\author{Group 1: Javier Esteban Aragoneses, Mauricio Marcos Fajgenbaun, Danyu
Zhang, Daniel Alonso}
\date{November 27th, 2020}

\begin{document}
\maketitle

Importing libraries

\begin{verbatim}
#> Package:  markovchain
#> Version:  0.8.5-2
#> Date:     2020-09-07
#> BugReport: https://github.com/spedygiorgio/markovchain/issues
#> 
#> Attaching package: 'dplyr'
#> The following objects are masked from 'package:stats':
#> 
#>     filter, lag
#> The following objects are masked from 'package:base':
#> 
#>     intersect, setdiff, setequal, union
\end{verbatim}

\hypertarget{problem-1}{%
\section{Problem 1}\label{problem-1}}

\begin{verbatim}
#> [1] "n bnq ndrsw awornoc aslnxs rzs dowfr ksfrswc ljicrwq rj nirzjwoms n ljwjcngowif gnllocs aworofz ncb nxswolnc jddolonhf aolpswsb jgsw kzolz tjgswcxscrf bwit nyywjgnh ywjlsff knf asrrsw hsnboct floscrofrf rj knwc rznr rzs bsanrs ljihb icbswxocs yiahol dnorz gnllocs cnrojcnhofx znf cj yhnls oc ljgob jw jrzsw yiahol zsnhrz xnrrswf jd thjanh fotcodolncls fnob n floscrodol nbgofsw rj rzs aworofz tjgswcxscr floscls znf nhknqf assc rzs suor frwnrstq dwjx rzof zjwwscbjif yncbsxol fsgswnh rjy aworofz hnkxnpswf zngs nhfj ocljwwslrhq lnfr rzs ljicrwqf fyhor korz rzs siwjysnc icojc nf rzs wsnfjc or nirzjwomsb n gnllocs dowfr oc dnlr awornoc wsxnocf icbsw rzs ahjlf wstihnrjwq ixawshhn ncb knf nahs rj xjgs xjws violphq aslnifs jd nc jhb hnk scnahoct or rj xnps orf jkc bsrswxocnrojcf oc yiahol zsnhrz sxswtsclosf"
\end{verbatim}

\hypertarget{problem-2}{%
\section{Problem 2}\label{problem-2}}

\hypertarget{a}{%
\section{(a)}\label{a}}

Let \textbf{\(N(t)\)} be the number of cars arriving at a parking lot by
time \textbf{\(t\)}, according to the proposed scenario, we can model
\textbf{\(N(t)\)} as a non-homogenous Poisson process. Such process has
almost the same process as any other Poisson process, however, its rate
is a function of time.

\(N(t), t \in [0, \infty)\) is the non-homogenous Poisson process with
rate \(\lambda (t)\) where:

\begin{itemize}
\tightlist
\item
  \(N(0) = 0\)
\item
  \(N(t)\) has independent increments
\end{itemize}

We define 8:00 as \(t=0\) with the following integrable function and
each unit of \(t\) equals to 1 hour:

\(\lambda (t) = \begin{cases} 100 & 0 \leq t \leq \frac{1}{2} \\ 600t - 200 & \frac{1}{2} < t \leq \frac{3}{4} \\ 400t - 50 & \frac{3}{4} < t \leq 1 \\ -500t + 850 & 1 < t \leq 1.5 \\ \end{cases}\)

So,

\(E[N(t)] = \begin{cases} \int_{0}^t 100\,dt = 100t & 0 \leq t \leq \frac{1}{2} \\ \int_{\frac{1}{2}}^t 600t - 200 \,dt + 50 = 300(t^2 - \frac{1}{4}) - 200(t - \frac{1}{2}) + 50 & \frac{1}{2} < t \leq \frac{3}{4} \\ \int_{\frac{3}{4}}^t 400t - 50 \,dt + 93.75 = 25(8t^2 - 2t - 3) + 93.75 & \frac{3}{4} < t \leq 1 \\ \int_{1}^t -500t + 850\,dt + 168.75 = -50(5t^2 - 17t + 12) + 168.75 & 1 < t \leq 1.5 \end{cases}\)

Given that there is a limit of 150 vehicles:

\(E[N(t)] = \begin{cases} 100t & 0 \leq t \leq \frac{1}{2} \\ 300(t^2 - \frac{1}{4}) - 200(t - \frac{1}{2}) + 50 & \frac{1}{2} < t \leq \frac{3}{4} \\ 25(8t^2 - 2t - 3) + 93.75 & \frac{3}{4} < t < 0.94468 \\ 150 & t \geq 0.94468 \end{cases}\)

\hypertarget{b}{%
\section{(b)}\label{b}}

\includegraphics{./figures/unnamed-chunk-4-1.pdf}
\includegraphics{./figures/unnamed-chunk-4-2.pdf}

Luego de hacer las pruebas para \(\lambda (t)\) obtenemos lo siguiente:

\begin{Shaded}
\begin{Highlighting}[]
\NormalTok{lambda =}\StringTok{ }\FloatTok{139.6}
\NormalTok{t =}\StringTok{ }\FloatTok{0.91232}
\CommentTok{# 8:44 AM}
\end{Highlighting}
\end{Shaded}

Por lo que \(t = 0.91232\) horas.

\hypertarget{c}{%
\section{(c)}\label{c}}

The following function simulates a non-homogenous poisson process from a
homogenous poisson process:

\begin{Shaded}
\begin{Highlighting}[]
\NormalTok{non_hom_poisson <-}\StringTok{ }\ControlFlowTok{function}\NormalTok{(fun,l,a,b,}\DataTypeTok{start=}\DecValTok{0}\NormalTok{) \{}
    \CommentTok{# This function generates a non-homogenous poisson }
    \CommentTok{# process from a homogenous poisson process}
    \CommentTok{# PARAMS:}
    \CommentTok{# fun:   if the non-homogenous poisson process has}
    \CommentTok{#        multiple functions per time subinterval}
    \CommentTok{#        this parameters represents such function}
    \CommentTok{# l:     lambda for the homogenous poisson process}
    \CommentTok{# a:     lower bound for the time subinterval}
    \CommentTok{# b:     upper bound for the time subinterval}
    \CommentTok{# start: this parameter is used to keep track of }
    \CommentTok{#        the process count.}
    
    \CommentTok{# We generate the homogenous poisson process }
    \CommentTok{# arrival times}
\NormalTok{    val <-}\StringTok{ }\KeywordTok{rpois}\NormalTok{(}\DecValTok{1}\NormalTok{,l}\OperatorTok{*}\NormalTok{(b}\OperatorTok{-}\NormalTok{a))}
\NormalTok{    intervals <-}\StringTok{ }\NormalTok{(b}\OperatorTok{-}\NormalTok{a) }\OperatorTok{*}\StringTok{ }\KeywordTok{sort}\NormalTok{(}\KeywordTok{runif}\NormalTok{(val)) }\OperatorTok{+}\StringTok{ }\NormalTok{a}

    \CommentTok{# Non-homogenous poisson process}
\NormalTok{    evs <-}\StringTok{ }\KeywordTok{length}\NormalTok{(intervals) }\CommentTok{# lenght of arrival times }
\NormalTok{    nh_val <-}\StringTok{ }\DecValTok{0} \OperatorTok{+}\StringTok{ }\NormalTok{start }\CommentTok{# start of the event count}
\NormalTok{    nh_intervals <-}\StringTok{ }\KeywordTok{c}\NormalTok{() }\CommentTok{# arrival times for the NHPP}
    \ControlFlowTok{for}\NormalTok{ (i }\ControlFlowTok{in} \DecValTok{1}\OperatorTok{:}\NormalTok{evs) \{}
        \ControlFlowTok{if}\NormalTok{ (}\KeywordTok{runif}\NormalTok{(}\DecValTok{1}\NormalTok{) }\OperatorTok{<}\StringTok{ }\KeywordTok{fun}\NormalTok{(intervals[i])}\OperatorTok{/}\NormalTok{l) \{ }
            \CommentTok{# only including intervals from the HPP which}
            \CommentTok{# match with fun(intervals[i])/l probability}
\NormalTok{            nh_intervals <-}\StringTok{ }\KeywordTok{c}\NormalTok{(nh_intervals, intervals[i]) }
\NormalTok{            nh_val <-}\StringTok{ }\NormalTok{nh_val}\OperatorTok{+}\DecValTok{1} \CommentTok{# adding one to the event count}
\NormalTok{        \}}
\NormalTok{    \}}
\NormalTok{    nh_events <-}\StringTok{ }\KeywordTok{seq}\NormalTok{(}\DecValTok{1}\OperatorTok{+}\NormalTok{start,nh_val,}\DecValTok{1}\NormalTok{) }\CommentTok{# events since the previous group}
    \KeywordTok{return}\NormalTok{(}\KeywordTok{list}\NormalTok{(}\DataTypeTok{arrival_times=}\NormalTok{nh_intervals, }\DataTypeTok{events=}\NormalTok{nh_events))}
\NormalTok{\}}
\end{Highlighting}
\end{Shaded}

\newpage

\begin{Shaded}
\begin{Highlighting}[]
\NormalTok{simulation <-}\StringTok{ }\ControlFlowTok{function}\NormalTok{(iters, functions, lambdas, ints) \{}
    \CommentTok{# This function simulates from the NHPP}
    \CommentTok{# iters:     number of iterations to plot and add to the list of}
    \CommentTok{#            dataframes}
    \CommentTok{# functions: list of functions corresponding to the lambda function}
    \CommentTok{# lambdas:   list of lambdas for each subinterval}
    \CommentTok{# ints:      lists of vectors of 2 elements each containing the intervals}
    \CommentTok{#            that correspond to each element of lambdas and functions lists}
\NormalTok{    p <-}\StringTok{ }\KeywordTok{list}\NormalTok{()}
    \ControlFlowTok{for}\NormalTok{ (i }\ControlFlowTok{in} \DecValTok{1}\OperatorTok{:}\NormalTok{iters) \{}
\NormalTok{        maximum <-}\StringTok{ }\DecValTok{0} \CommentTok{# start for the next NHPP simulation to continue count}
\NormalTok{        arr_times <-}\StringTok{ }\KeywordTok{c}\NormalTok{() }\CommentTok{# arrival times}
\NormalTok{        events <-}\StringTok{ }\KeywordTok{c}\NormalTok{() }\CommentTok{# event counts}
        \ControlFlowTok{for}\NormalTok{ (k }\ControlFlowTok{in} \DecValTok{1}\OperatorTok{:}\DecValTok{4}\NormalTok{) \{}
\NormalTok{            int <-}\StringTok{ }\KeywordTok{non_hom_poisson}\NormalTok{(lambda_funs[[k]],lambdas[[k]],}
\NormalTok{                                   ints[[k]][}\DecValTok{1}\NormalTok{],ints[[k]][}\DecValTok{2}\NormalTok{],}
                                   \DataTypeTok{start=}\NormalTok{maximum)}
\NormalTok{            maximum <-}\StringTok{ }\KeywordTok{max}\NormalTok{(int}\OperatorTok{$}\NormalTok{events) }\CommentTok{# remembering last event count}
\NormalTok{            arr_times <-}\StringTok{ }\KeywordTok{c}\NormalTok{(arr_times, int}\OperatorTok{$}\NormalTok{arrival_times) }
\NormalTok{            events <-}\StringTok{ }\KeywordTok{c}\NormalTok{(events, int}\OperatorTok{$}\NormalTok{events)}
\NormalTok{        \}}
\NormalTok{        p[[i]] <-}\StringTok{ }\KeywordTok{data.frame}\NormalTok{(}\DataTypeTok{arrival_times=}\NormalTok{arr_times, }\DataTypeTok{events=}\NormalTok{events)}
        \CommentTok{# plots}
        \ControlFlowTok{if}\NormalTok{ (i }\OperatorTok{==}\StringTok{ }\DecValTok{1}\NormalTok{) \{}\KeywordTok{plot}\NormalTok{(arr_times, events, }\DataTypeTok{cex=}\FloatTok{0.5}\NormalTok{, }\DataTypeTok{pch=}\StringTok{'.'}\NormalTok{, }
                          \DataTypeTok{col=}\KeywordTok{randomColor}\NormalTok{(), }\DataTypeTok{xlim=}\KeywordTok{c}\NormalTok{(}\DecValTok{0}\NormalTok{,}\FloatTok{1.5}\NormalTok{), }
                          \DataTypeTok{ylim=}\KeywordTok{c}\NormalTok{(}\DecValTok{0}\NormalTok{,}\DecValTok{150}\NormalTok{))\}}
        \KeywordTok{lines}\NormalTok{(arr_times, events, }\DataTypeTok{col=}\KeywordTok{randomColor}\NormalTok{())}
\NormalTok{    \}}
    \KeywordTok{return}\NormalTok{(p)}
\NormalTok{\}}
\end{Highlighting}
\end{Shaded}

\newpage

\begin{Shaded}
\begin{Highlighting}[]
\NormalTok{data <-}\StringTok{ }\KeywordTok{simulation}\NormalTok{(}\DecValTok{1000}\NormalTok{, lambda_funs, lambdas, ints)}
\end{Highlighting}
\end{Shaded}

\includegraphics{./figures/unnamed-chunk-9-1.pdf}

\begin{Shaded}
\begin{Highlighting}[]
\NormalTok{ratio <-}\StringTok{ }\DecValTok{0}
\ControlFlowTok{for}\NormalTok{ (i }\ControlFlowTok{in} \DecValTok{1}\OperatorTok{:}\KeywordTok{length}\NormalTok{(data)) \{}
\NormalTok{    df <-}\StringTok{ }\KeywordTok{data.frame}\NormalTok{(data[[i]])}
\NormalTok{    cnt <-}\StringTok{ }\NormalTok{df }\OperatorTok\StringTok{ }\KeywordTok{filter}\NormalTok{(arrival_times }\OperatorTok{<}\StringTok{ }\FloatTok{0.91232} \OperatorTok{&}\StringTok{ }\NormalTok{events }\OperatorTok{>=}\StringTok{ }\DecValTok{150}\NormalTok{) }\OperatorTok\StringTok{ }\NormalTok{dplyr}\OperatorTok{::}\KeywordTok{count}\NormalTok{()}
    \ControlFlowTok{if}\NormalTok{ (cnt[}\DecValTok{1}\NormalTok{] }\OperatorTok{>=}\StringTok{ }\DecValTok{1}\NormalTok{) \{}
\NormalTok{        ratio <-}\StringTok{ }\NormalTok{ratio }\OperatorTok{+}\StringTok{ }\DecValTok{1}
\NormalTok{    \}}
\NormalTok{\}}
\NormalTok{ratio}\OperatorTok{/}\DecValTok{1000}
\CommentTok{#> [1] 0.097}
\end{Highlighting}
\end{Shaded}

\hypertarget{problem-3}{%
\section{Problem 3}\label{problem-3}}

\hypertarget{a-1}{%
\section{(a)}\label{a-1}}

Our infinitesimal generator is the following:

\(Q = \begin{pmatrix} - \lambda & \lambda & 0 & 0 & \dots \\ \mu & - (\lambda + \mu) & \lambda & 0 & \dots \\ 0 & 2 \mu & - (2 \mu + \lambda) & \lambda & \dots \\ \vdots & \vdots & \vdots & \vdots & \ddots \end{pmatrix}\)

\hypertarget{b-1}{%
\section{(b)}\label{b-1}}

We solve the following system:

\(\begin{cases} \sum_{i = 1}^{\infty} \pi_i = 1 \\ \lambda \pi_0 = \mu \pi_1 \\ \lambda \pi_1 = 2 \mu \pi_2 \\ \vdots \\ \lambda \pi_{n-1} = 2 \mu \pi_n \\ \vdots \\ \end{cases}\)

First we have:

\(\pi_1 = \frac{\lambda \pi_0}{\mu} \\ \pi_2 = \frac{\lambda^2 \pi_0}{2 \mu^2} \\ \pi_3 = \frac{\lambda^3 \pi_0}{2^2 \mu^3} \\ \vdots \\ \pi_n = \frac{\lambda^n \pi_0}{2^{n-1} \mu^n} \\ \vdots\)

Then:

\(\sum_{i=0}^{\infty} \pi_i = \pi_0 + \frac{\lambda \pi_0}{\mu} + \frac{\lambda^2 \pi_0}{2 \mu^2} + \dots + \frac{\lambda^n \pi_0}{2^{n-1} \mu^n} + \dots = 1\)

And so factoring \(\pi_0\) we get:

\(\pi_0 (1 + \frac{\lambda}{\mu} + \frac{\lambda^2}{2 \mu^2} + \dots) = \pi_0 (1 + \sum_{i=1}^{\infty} \frac{1}{2^{i-1}} (\frac{\lambda}{\mu})^{i})\)

Then multiplying \(\frac{2}{2}\) to the summation:

\(\pi_0 (1 + \frac{2}{2} \sum_{i=1}^{\infty} \frac{1}{2^{i-1}} (\frac{\lambda}{\mu})^{i})\)

= \(\pi_0 (2 \sum_{i=0}^{\infty} (\frac{\lambda}{2 \mu})^{i} - 1)\)

\(\pi_0 (2 (\frac{1}{1 - \frac{\lambda}{2 \mu}}) - 1) = 1\)

\(\pi_0 = \frac{1}{2 (\frac{1}{1 - \frac{\lambda}{2 \mu}}) - 1}\)

\(\vdots\)

\(\pi_n = \frac{\lambda^n}{2^{n-1}{\mu_n} \pi_0} \frac{1}{2 (\frac{1}{1 - \frac{\lambda}{2 \mu}}) - 1}\)

finally:

\(\pi_n = \frac{1}{2^{n-1}} (\frac{\lambda}{\mu})^{n} \pi_0\)

\end{document}
