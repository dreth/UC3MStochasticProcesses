\PassOptionsToPackage{unicode=true}{hyperref} % options for packages loaded elsewhere
\PassOptionsToPackage{hyphens}{url}
%
\documentclass[]{article}
\usepackage{lmodern}
\usepackage{amssymb,amsmath}
\usepackage{ifxetex,ifluatex}
\usepackage{fixltx2e} % provides \textsubscript
\ifnum 0\ifxetex 1\fi\ifluatex 1\fi=0 % if pdftex
  \usepackage[T1]{fontenc}
  \usepackage[utf8]{inputenc}
  \usepackage{textcomp} % provides euro and other symbols
\else % if luatex or xelatex
  \usepackage{unicode-math}
  \defaultfontfeatures{Ligatures=TeX,Scale=MatchLowercase}
\fi
% use upquote if available, for straight quotes in verbatim environments
\IfFileExists{upquote.sty}{\usepackage{upquote}}{}
% use microtype if available
\IfFileExists{microtype.sty}{%
\usepackage[]{microtype}
\UseMicrotypeSet[protrusion]{basicmath} % disable protrusion for tt fonts
}{}
\IfFileExists{parskip.sty}{%
\usepackage{parskip}
}{% else
\setlength{\parindent}{0pt}
\setlength{\parskip}{6pt plus 2pt minus 1pt}
}
\usepackage{hyperref}
\hypersetup{
            pdftitle={Stochastic Processes: Assignment 1},
            pdfauthor={Group 1: Javier Esteban Aragoneses, Mauricio Marcos Fajgenbaun, Danyu Zhang, Daniel Alonso},
            pdfborder={0 0 0},
            breaklinks=true}
\urlstyle{same}  % don't use monospace font for urls
\usepackage[margin=1in]{geometry}
\usepackage{color}
\usepackage{fancyvrb}
\newcommand{\VerbBar}{|}
\newcommand{\VERB}{\Verb[commandchars=\\\{\}]}
\DefineVerbatimEnvironment{Highlighting}{Verbatim}{commandchars=\\\{\}}
% Add ',fontsize=\small' for more characters per line
\usepackage{framed}
\definecolor{shadecolor}{RGB}{248,248,248}
\newenvironment{Shaded}{\begin{snugshade}}{\end{snugshade}}
\newcommand{\AlertTok}[1]{\textcolor[rgb]{0.94,0.16,0.16}{#1}}
\newcommand{\AnnotationTok}[1]{\textcolor[rgb]{0.56,0.35,0.01}{\textbf{\textit{#1}}}}
\newcommand{\AttributeTok}[1]{\textcolor[rgb]{0.77,0.63,0.00}{#1}}
\newcommand{\BaseNTok}[1]{\textcolor[rgb]{0.00,0.00,0.81}{#1}}
\newcommand{\BuiltInTok}[1]{#1}
\newcommand{\CharTok}[1]{\textcolor[rgb]{0.31,0.60,0.02}{#1}}
\newcommand{\CommentTok}[1]{\textcolor[rgb]{0.56,0.35,0.01}{\textit{#1}}}
\newcommand{\CommentVarTok}[1]{\textcolor[rgb]{0.56,0.35,0.01}{\textbf{\textit{#1}}}}
\newcommand{\ConstantTok}[1]{\textcolor[rgb]{0.00,0.00,0.00}{#1}}
\newcommand{\ControlFlowTok}[1]{\textcolor[rgb]{0.13,0.29,0.53}{\textbf{#1}}}
\newcommand{\DataTypeTok}[1]{\textcolor[rgb]{0.13,0.29,0.53}{#1}}
\newcommand{\DecValTok}[1]{\textcolor[rgb]{0.00,0.00,0.81}{#1}}
\newcommand{\DocumentationTok}[1]{\textcolor[rgb]{0.56,0.35,0.01}{\textbf{\textit{#1}}}}
\newcommand{\ErrorTok}[1]{\textcolor[rgb]{0.64,0.00,0.00}{\textbf{#1}}}
\newcommand{\ExtensionTok}[1]{#1}
\newcommand{\FloatTok}[1]{\textcolor[rgb]{0.00,0.00,0.81}{#1}}
\newcommand{\FunctionTok}[1]{\textcolor[rgb]{0.00,0.00,0.00}{#1}}
\newcommand{\ImportTok}[1]{#1}
\newcommand{\InformationTok}[1]{\textcolor[rgb]{0.56,0.35,0.01}{\textbf{\textit{#1}}}}
\newcommand{\KeywordTok}[1]{\textcolor[rgb]{0.13,0.29,0.53}{\textbf{#1}}}
\newcommand{\NormalTok}[1]{#1}
\newcommand{\OperatorTok}[1]{\textcolor[rgb]{0.81,0.36,0.00}{\textbf{#1}}}
\newcommand{\OtherTok}[1]{\textcolor[rgb]{0.56,0.35,0.01}{#1}}
\newcommand{\PreprocessorTok}[1]{\textcolor[rgb]{0.56,0.35,0.01}{\textit{#1}}}
\newcommand{\RegionMarkerTok}[1]{#1}
\newcommand{\SpecialCharTok}[1]{\textcolor[rgb]{0.00,0.00,0.00}{#1}}
\newcommand{\SpecialStringTok}[1]{\textcolor[rgb]{0.31,0.60,0.02}{#1}}
\newcommand{\StringTok}[1]{\textcolor[rgb]{0.31,0.60,0.02}{#1}}
\newcommand{\VariableTok}[1]{\textcolor[rgb]{0.00,0.00,0.00}{#1}}
\newcommand{\VerbatimStringTok}[1]{\textcolor[rgb]{0.31,0.60,0.02}{#1}}
\newcommand{\WarningTok}[1]{\textcolor[rgb]{0.56,0.35,0.01}{\textbf{\textit{#1}}}}
\usepackage{graphicx,grffile}
\makeatletter
\def\maxwidth{\ifdim\Gin@nat@width>\linewidth\linewidth\else\Gin@nat@width\fi}
\def\maxheight{\ifdim\Gin@nat@height>\textheight\textheight\else\Gin@nat@height\fi}
\makeatother
% Scale images if necessary, so that they will not overflow the page
% margins by default, and it is still possible to overwrite the defaults
% using explicit options in \includegraphics[width, height, ...]{}
\setkeys{Gin}{width=\maxwidth,height=\maxheight,keepaspectratio}
\setlength{\emergencystretch}{3em}  % prevent overfull lines
\providecommand{\tightlist}{%
  \setlength{\itemsep}{0pt}\setlength{\parskip}{0pt}}
\setcounter{secnumdepth}{0}
% Redefines (sub)paragraphs to behave more like sections
\ifx\paragraph\undefined\else
\let\oldparagraph\paragraph
\renewcommand{\paragraph}[1]{\oldparagraph{#1}\mbox{}}
\fi
\ifx\subparagraph\undefined\else
\let\oldsubparagraph\subparagraph
\renewcommand{\subparagraph}[1]{\oldsubparagraph{#1}\mbox{}}
\fi

% set default figure placement to htbp
\makeatletter
\def\fps@figure{htbp}
\makeatother


\title{Stochastic Processes: Assignment 1}
\author{Group 1: Javier Esteban Aragoneses, Mauricio Marcos Fajgenbaun, Danyu
Zhang, Daniel Alonso}
\date{January 10th, 2020}

\begin{document}
\maketitle

Importing libraries

\begin{verbatim}
#> Package:  markovchain
#> Version:  0.8.5-2
#> Date:     2020-09-07
#> BugReport: https://github.com/spedygiorgio/markovchain/issues
#> 
#> Attaching package: 'dplyr'
#> The following objects are masked from 'package:stats':
#> 
#>     filter, lag
#> The following objects are masked from 'package:base':
#> 
#>     intersect, setdiff, setequal, union
\end{verbatim}

\hypertarget{problem-1}{%
\section{Problem 1}\label{problem-1}}

\begin{Shaded}
\begin{Highlighting}[]
\CommentTok{# importing libraries}
\ImportTok{import}\NormalTok{ numpy }\ImportTok{as}\NormalTok{ np}
\ImportTok{from}\NormalTok{ copy }\ImportTok{import}\NormalTok{ deepcopy}
\ImportTok{import}\NormalTok{ pandas }\ImportTok{as}\NormalTok{ pd}
\ImportTok{import}\NormalTok{ matplotlib.pyplot }\ImportTok{as}\NormalTok{ plt}
\ImportTok{from}\NormalTok{ nltk.corpus }\ImportTok{import}\NormalTok{ words}

\CommentTok{# Importing the matrix with the frequencies}
\NormalTok{freq }\OperatorTok{=}\NormalTok{  np.loadtxt(}\StringTok{'./data/Englishcharacters.txt'}\NormalTok{, usecols}\OperatorTok{=}\BuiltInTok{range}\NormalTok{(}\DecValTok{27}\NormalTok{))}

\CommentTok{# transformation function for table values}
\KeywordTok{def}\NormalTok{ transf_log(table):}
    \ControlFlowTok{return}\NormalTok{ np.log(table }\OperatorTok{+} \DecValTok{1}\NormalTok{)}

\CommentTok{# applying the transformation function}
\NormalTok{freq }\OperatorTok{=}\NormalTok{ transf_log(freq)}

\CommentTok{# importing the messages and only selecting the }
\CommentTok{# message with index 0, as this is the one corresponding}
\CommentTok{# to group 1}
\ControlFlowTok{with} \BuiltInTok{open}\NormalTok{(}\StringTok{'./data/messages.txt'}\NormalTok{) }\ImportTok{as}\NormalTok{ f:}
\NormalTok{    message }\OperatorTok{=}\NormalTok{ f.readlines()[}\DecValTok{0}\NormalTok{].replace(}\StringTok{'}\CharTok{\textbackslash{}n}\StringTok{'}\NormalTok{,}\StringTok{''}\NormalTok{)}

\CommentTok{# decode function}
\KeywordTok{def}\NormalTok{ decode(message, freqs, iters):}
    \CommentTok{"""}
\CommentTok{    Function to decode a message enconded }
\CommentTok{    using a substitution cipher utilizing the}
\CommentTok{    Metropolis-Hastings algorithm.}

\CommentTok{    Params:}
\CommentTok{    message = string to decode}
\CommentTok{    freqs = frequency table for transitions of letters}
\CommentTok{            to input}
\CommentTok{    iters = amount of iterations to perform}
\CommentTok{    """}
    \CommentTok{# dictionary to organize the iterations}
    \CommentTok{# the score and the result of the attempt}
    \CommentTok{# to decode the message corresponding to that}
    \CommentTok{# iteration}
\NormalTok{    msg_iters }\OperatorTok{=}\NormalTok{ \{\}}

    \CommentTok{# defining the identity function}
    \CommentTok{# all letters to be used excluding spaces}
\NormalTok{    letters }\OperatorTok{=}\NormalTok{ [}\StringTok{"a"}\NormalTok{, }\StringTok{"b"}\NormalTok{, }\StringTok{"c"}\NormalTok{, }\StringTok{"d"}\NormalTok{,}
                \StringTok{"e"}\NormalTok{, }\StringTok{"f"}\NormalTok{, }\StringTok{"g"}\NormalTok{, }\StringTok{"h"}\NormalTok{, }
                \StringTok{"i"}\NormalTok{, }\StringTok{"j"}\NormalTok{, }\StringTok{"k"}\NormalTok{, }\StringTok{"l"}\NormalTok{, }
                \StringTok{"m"}\NormalTok{, }\StringTok{"n"}\NormalTok{, }\StringTok{"o"}\NormalTok{, }\StringTok{"p"}\NormalTok{, }
                \StringTok{"q"}\NormalTok{, }\StringTok{"r"}\NormalTok{, }\StringTok{"s"}\NormalTok{, }\StringTok{"t"}\NormalTok{, }
                \StringTok{"u"}\NormalTok{, }\StringTok{"v"}\NormalTok{, }\StringTok{"w"}\NormalTok{, }\StringTok{"x"}\NormalTok{, }
                \StringTok{"y"}\NormalTok{, }\StringTok{"z"}\NormalTok{]}
    \CommentTok{# creating a copy of the original letters}
    \CommentTok{# to use as key for the dictionaries}
\NormalTok{    init_letters }\OperatorTok{=}\NormalTok{ deepcopy(letters)}
    \CommentTok{# creating a dictionary with init_letters as keys}
    \CommentTok{# and letters as values}
\NormalTok{    cd }\OperatorTok{=}\NormalTok{ \{l:d }\ControlFlowTok{for}\NormalTok{ l,d }\KeywordTok{in} \BuiltInTok{zip}\NormalTok{(init_letters,letters)\}}
    \CommentTok{# every time we update with \{' ':' '\} we add the space}
    \CommentTok{# to the dictionary}
\NormalTok{    cd.update(\{}\StringTok{' '}\NormalTok{:}\StringTok{' '}\NormalTok{\})}
    \CommentTok{# define a function that just uses the previous}
    \CommentTok{# dictionary to seek the letters}
    \KeywordTok{def}\NormalTok{ f(c):}
        \ControlFlowTok{return}\NormalTok{ cd[c]}
    
    \CommentTok{# this dictionary and subsequent function maps each letter}
    \CommentTok{# to a column/row in the freq matrix, ex: 'a':0, 'b':1 }
\NormalTok{    fvals }\OperatorTok{=}\NormalTok{ np.array([x }\ControlFlowTok{for}\NormalTok{ x }\KeywordTok{in} \BuiltInTok{range}\NormalTok{(}\BuiltInTok{len}\NormalTok{(letters))])}
\NormalTok{    cd_map }\OperatorTok{=}\NormalTok{ \{l:v }\ControlFlowTok{for}\NormalTok{ l,v }\KeywordTok{in} \BuiltInTok{zip}\NormalTok{(letters,fvals)\}}
\NormalTok{    cd_map.update(\{}\StringTok{' '}\NormalTok{:}\DecValTok{26}\NormalTok{\})}
    \KeywordTok{def}\NormalTok{ f_map(c):}
        \ControlFlowTok{return}\NormalTok{ cd_map[c]}

    \CommentTok{# score function uses sum of logs}
    \KeywordTok{def}\NormalTok{ score(fun):}
\NormalTok{        p }\OperatorTok{=} \DecValTok{0}
        \ControlFlowTok{for}\NormalTok{ i }\KeywordTok{in} \BuiltInTok{range}\NormalTok{(}\DecValTok{1}\NormalTok{,}\BuiltInTok{len}\NormalTok{(msg)):}
\NormalTok{            p }\OperatorTok{=}\NormalTok{ p }\OperatorTok{+}\NormalTok{ freq[f_map(fun(msg[i}\DecValTok{-1}\NormalTok{])),f_map(fun(msg[i]))]}
        \ControlFlowTok{return}\NormalTok{ p}
    
    \CommentTok{# converting the message to a list in order to}
    \CommentTok{# go through the letters in pairs}
\NormalTok{    msg }\OperatorTok{=} \BuiltInTok{list}\NormalTok{(message)}

    \CommentTok{# letters list, this one shall be modified}
    \CommentTok{# every time the score passes the test}
\NormalTok{    letters_n }\OperatorTok{=}\NormalTok{ deepcopy(letters)}

    \CommentTok{# loop iters amount of times}
    \ControlFlowTok{for}\NormalTok{ i }\KeywordTok{in} \BuiltInTok{range}\NormalTok{(iters):}
        \CommentTok{# randomly choose 2 numbers and replace the 2 chosen }
        \CommentTok{# vals in a copy of letters}
\NormalTok{        ch1 }\OperatorTok{=}\NormalTok{ np.random.randint(}\DecValTok{0}\NormalTok{,}\BuiltInTok{len}\NormalTok{(letters))}
\NormalTok{        ch2 }\OperatorTok{=}\NormalTok{ np.random.randint(}\DecValTok{0}\NormalTok{,}\BuiltInTok{len}\NormalTok{(letters))}
\NormalTok{        plc1 }\OperatorTok{=}\NormalTok{ deepcopy(letters_n[ch1])}
\NormalTok{        plc2 }\OperatorTok{=}\NormalTok{ deepcopy(letters_n[ch2])}
\NormalTok{        letters_n[ch1] }\OperatorTok{=}\NormalTok{ plc2}
\NormalTok{        letters_n[ch2] }\OperatorTok{=}\NormalTok{ plc1}

        \CommentTok{# create the dictionary for the f* function}
\NormalTok{        cd_n }\OperatorTok{=}\NormalTok{ \{l:v }\ControlFlowTok{for}\NormalTok{ l,v }\KeywordTok{in} \BuiltInTok{zip}\NormalTok{(init_letters,letters_n)\}}
        \CommentTok{# add the space to it after scramble}
\NormalTok{        cd_n.update(\{}\StringTok{' '}\NormalTok{:}\StringTok{' '}\NormalTok{\})}
        \CommentTok{# f* definition}
        \KeywordTok{def}\NormalTok{ f_n(c):}
            \ControlFlowTok{return}\NormalTok{ cd_n[c]}
    
        \CommentTok{# calculating the score for each function and its ratio}
\NormalTok{        scr_f }\OperatorTok{=}\NormalTok{ score(f)}
\NormalTok{        scr_fn }\OperatorTok{=}\NormalTok{ score(f_n)}
\NormalTok{        a }\OperatorTok{=}\NormalTok{ scr_fn}\OperatorTok{/}\NormalTok{scr_f}
        \CommentTok{# test if a random number is lower than min(a, 1)}
\NormalTok{        cond }\OperatorTok{=}\NormalTok{ np.random.rand() }\OperatorTok{<=} \BuiltInTok{min}\NormalTok{(a,}\DecValTok{1}\NormalTok{)}
        \CommentTok{# if condition is true}
        \ControlFlowTok{if}\NormalTok{ cond:}
            \CommentTok{# replacing the letters list with the one from f*}
\NormalTok{            letters }\OperatorTok{=}\NormalTok{ deepcopy(letters_n)}
            \CommentTok{# updating the dictionary with the new letters list after replacing}
\NormalTok{            cd }\OperatorTok{=}\NormalTok{ \{l:v }\ControlFlowTok{for}\NormalTok{ l,v }\KeywordTok{in} \BuiltInTok{zip}\NormalTok{(init_letters,letters)\}}
\NormalTok{            cd.update(\{}\StringTok{' '}\NormalTok{:}\StringTok{' '}\NormalTok{\})}
            \CommentTok{# f re-definition}
            \KeywordTok{def}\NormalTok{ f(c):}
                \ControlFlowTok{return}\NormalTok{ cd[c]}
            \CommentTok{# replacing the letters in the message using }
            \CommentTok{# the new f replaced by f*}
            \ControlFlowTok{for}\NormalTok{ k }\KeywordTok{in} \BuiltInTok{range}\NormalTok{(}\BuiltInTok{len}\NormalTok{(msg)):}
\NormalTok{                msg[k] }\OperatorTok{=}\NormalTok{ f(msg[k])}
            \CommentTok{# adding score and joining the message to the dictionary}
            \CommentTok{# to then transform into a dataframe}
\NormalTok{            msg_iters[i] }\OperatorTok{=}\NormalTok{ (a,}\StringTok{''}\NormalTok{.join([x }\ControlFlowTok{for}\NormalTok{ x }\KeywordTok{in}\NormalTok{ msg]))}
        \CommentTok{# if condition is false}
        \ControlFlowTok{else}\NormalTok{:}
            \CommentTok{# apply the f function to the message instead}
            \ControlFlowTok{for}\NormalTok{ k }\KeywordTok{in} \BuiltInTok{range}\NormalTok{(}\BuiltInTok{len}\NormalTok{(msg)):}
\NormalTok{                msg[k] }\OperatorTok{=}\NormalTok{ f(msg[k])}
        \CommentTok{# reset the message}
\NormalTok{        msg }\OperatorTok{=} \BuiltInTok{list}\NormalTok{(message)}

        \CommentTok{# break the loop if 2 words are found in the english language corpus}
        \ControlFlowTok{try}\NormalTok{:}
\NormalTok{            msg_list }\OperatorTok{=}\NormalTok{ msg_iters[i].split(}\StringTok{' '}\NormalTok{)}
\NormalTok{            fw }\OperatorTok{=}\NormalTok{ \{}\StringTok{'w1'}\NormalTok{:np.random.choice(msg_list),}
                  \StringTok{'w2'}\NormalTok{:np.random.choice(msg_list)\}}
\NormalTok{            conds }\OperatorTok{=}\NormalTok{ \{w_n:(w }\KeywordTok{in}\NormalTok{ words.words()) }\ControlFlowTok{for}\NormalTok{ w_n,w }\KeywordTok{in}\NormalTok{ fw.items()\}}
\NormalTok{            vals }\OperatorTok{=} \BuiltInTok{list}\NormalTok{(conds.values())}
            \ControlFlowTok{if} \VariableTok{False} \KeywordTok{not} \KeywordTok{in}\NormalTok{ vals:}
                \BuiltInTok{print}\NormalTok{(}\SpecialStringTok{f'found at iteration: }\SpecialCharTok{\{i\}}\SpecialStringTok{'}\NormalTok{)}
                \BuiltInTok{print}\NormalTok{(msg_iters[i])}
                \BuiltInTok{print}\NormalTok{(}\SpecialStringTok{f'words found: }\SpecialCharTok{\{}\BuiltInTok{list}\NormalTok{(conds.keys())}\SpecialCharTok{\}}\SpecialStringTok{'}\NormalTok{)}
                \ControlFlowTok{break}
        \CommentTok{# otherwise continue}
        \ControlFlowTok{except}\NormalTok{:}
            \ControlFlowTok{continue}
    \CommentTok{# put the information in a dataframe, iters, score and the messages}
\NormalTok{    df }\OperatorTok{=}\NormalTok{ \{}\StringTok{'iter'}\NormalTok{:[it }\ControlFlowTok{for}\NormalTok{ it }\KeywordTok{in}\NormalTok{ msg_iters.keys()],}
          \StringTok{'score'}\NormalTok{:[msg[}\DecValTok{0}\NormalTok{] }\ControlFlowTok{for}\NormalTok{ msg }\KeywordTok{in}\NormalTok{ msg_iters.values()],}
          \StringTok{'msg'}\NormalTok{:[msg[}\DecValTok{1}\NormalTok{] }\ControlFlowTok{for}\NormalTok{ msg }\KeywordTok{in}\NormalTok{ msg_iters.values()]\}}
    \CommentTok{# return the dataframe}
    \ControlFlowTok{return}\NormalTok{ pd.DataFrame(df)}

\CommentTok{# we run the function}
\NormalTok{result }\OperatorTok{=}\NormalTok{ decode(message,freq, }\DecValTok{50000}\NormalTok{)}
\NormalTok{plt.plot(result.sort_values(}\StringTok{'score'}\NormalTok{)[}\StringTok{'score'}\NormalTok{].reset_index(drop}\OperatorTok{=}\VariableTok{True}\NormalTok{))}
\NormalTok{plt.show()}
\end{Highlighting}
\end{Shaded}

\includegraphics{./figures/unnamed-chunk-2-1.pdf}

\begin{Shaded}
\begin{Highlighting}[]
\BuiltInTok{print}\NormalTok{(result[result[}\StringTok{'score'}\NormalTok{] }\OperatorTok{==} \BuiltInTok{max}\NormalTok{(result[}\StringTok{'score'}\NormalTok{])])}
\CommentTok{#>         iter     score                                                msg}
\CommentTok{#> 30938  31353  1.319558  v svt vwfoc hcqfvqp hozvdo flo wqcbf mobfocp z...}
\end{Highlighting}
\end{Shaded}

\begin{verbatim}
#> [1] "n bnq ndrsw awornoc aslnxs rzs dowfr ksfrswc ljicrwq rj nirzjwoms n ljwjcngowif gnllocs aworofz ncb nxswolnc jddolonhf aolpswsb jgsw kzolz tjgswcxscrf bwit nyywjgnh ywjlsff knf asrrsw hsnboct floscrofrf rj knwc rznr rzs bsanrs ljihb icbswxocs yiahol dnorz gnllocs cnrojcnhofx znf cj yhnls oc ljgob jw jrzsw yiahol zsnhrz xnrrswf jd thjanh fotcodolncls fnob n floscrodol nbgofsw rj rzs aworofz tjgswcxscr floscls znf nhknqf assc rzs suor frwnrstq dwjx rzof zjwwscbjif yncbsxol fsgswnh rjy aworofz hnkxnpswf zngs nhfj ocljwwslrhq lnfr rzs ljicrwqf fyhor korz rzs siwjysnc icojc nf rzs wsnfjc or nirzjwomsb n gnllocs dowfr oc dnlr awornoc wsxnocf icbsw rzs ahjlf wstihnrjwq ixawshhn ncb knf nahs rj xjgs xjws violphq aslnifs jd nc jhb hnk scnahoct or rj xnps orf jkc bsrswxocnrojcf oc yiahol zsnhrz sxswtsclosf"
\end{verbatim}

\hypertarget{problem-2}{%
\section{Problem 2}\label{problem-2}}

\hypertarget{a}{%
\section{(a)}\label{a}}

Let \textbf{\(N(t)\)} be the number of cars arriving at a parking lot by
time \textbf{\(t\)}, according to the proposed scenario, we can model
\textbf{\(N(t)\)} as a non-homogenous Poisson process. Such process has
almost the same process as any other Poisson process, however, its rate
is a function of time.

\(N(t), t \in [0, \infty)\) is the non-homogenous Poisson process with
rate \(\lambda (t)\) where:

\begin{itemize}
\tightlist
\item
  \(N(0) = 0\)
\item
  \(N(t)\) has independent increments
\end{itemize}

We define 8:00 as \(t=0\) with the following integrable function and
each unit of \(t\) equals to 1 hour:

\(\lambda (t) = \begin{cases} 100 & 0 \leq t \leq \frac{1}{2} \\ 600t - 200 & \frac{1}{2} < t \leq \frac{3}{4} \\ 400t - 50 & \frac{3}{4} < t \leq 1 \\ -500t + 850 & 1 < t \leq 1.5 \\ \end{cases}\)

So,

\(E[N(t)] = \begin{cases} \int_{0}^t 100\,dt = 100t & 0 \leq t \leq \frac{1}{2} \\ \int_{\frac{1}{2}}^t 600t - 200 \,dt + 50 = 300(t^2 - \frac{1}{4}) - 200(t - \frac{1}{2}) + 50 & \frac{1}{2} < t \leq \frac{3}{4} \\ \int_{\frac{3}{4}}^t 400t - 50 \,dt + 93.75 = 25(8t^2 - 2t - 3) + 93.75 & \frac{3}{4} < t \leq 1 \\ \int_{1}^t -500t + 850\,dt + 168.75 = -50(5t^2 - 17t + 12) + 168.75 & 1 < t \leq 1.5 \end{cases}\)

Given that there is a limit of 150 vehicles:

\(E[N(t)] = \begin{cases} 100t & 0 \leq t \leq \frac{1}{2} \\ 300(t^2 - \frac{1}{4}) - 200(t - \frac{1}{2}) + 50 & \frac{1}{2} < t \leq \frac{3}{4} \\ 25(8t^2 - 2t - 3) + 93.75 & \frac{3}{4} < t < 0.94468 \\ 150 & t \geq 0.94468 \end{cases}\)

\hypertarget{b}{%
\section{(b)}\label{b}}

\includegraphics{./figures/unnamed-chunk-4-1.pdf}
\includegraphics{./figures/unnamed-chunk-4-2.pdf}

Luego de hacer las pruebas para \(\lambda (t)\) obtenemos lo siguiente:

\begin{Shaded}
\begin{Highlighting}[]
\NormalTok{lambda =}\StringTok{ }\FloatTok{139.6}
\NormalTok{t =}\StringTok{ }\FloatTok{0.91232}
\CommentTok{# 8:44 AM}
\end{Highlighting}
\end{Shaded}

Por lo que \(t = 0.91232\) horas (aproximadamente a las 8:54 de la
mañana).

\hypertarget{c}{%
\section{(c)}\label{c}}

The following function simulates a non-homogenous poisson process from a
homogenous poisson process:

\begin{Shaded}
\begin{Highlighting}[]
\NormalTok{non_hom_poisson <-}\StringTok{ }\ControlFlowTok{function}\NormalTok{(fun,l,a,b,}\DataTypeTok{start=}\DecValTok{0}\NormalTok{) \{}
    \CommentTok{# This function generates a non-homogenous poisson }
    \CommentTok{# process from a homogenous poisson process}
    \CommentTok{# PARAMS:}
    \CommentTok{# fun:   if the non-homogenous poisson process has}
    \CommentTok{#        multiple functions per time subinterval}
    \CommentTok{#        this parameters represents such function}
    \CommentTok{# l:     lambda for the homogenous poisson process}
    \CommentTok{# a:     lower bound for the time subinterval}
    \CommentTok{# b:     upper bound for the time subinterval}
    \CommentTok{# start: this parameter is used to keep track of }
    \CommentTok{#        the process count.}
    
    \CommentTok{# We generate the homogenous poisson process }
    \CommentTok{# arrival times}
\NormalTok{    val <-}\StringTok{ }\KeywordTok{rpois}\NormalTok{(}\DecValTok{1}\NormalTok{,l}\OperatorTok{*}\NormalTok{(b}\OperatorTok{-}\NormalTok{a))}
\NormalTok{    intervals <-}\StringTok{ }\NormalTok{(b}\OperatorTok{-}\NormalTok{a) }\OperatorTok{*}\StringTok{ }\KeywordTok{sort}\NormalTok{(}\KeywordTok{runif}\NormalTok{(val)) }\OperatorTok{+}\StringTok{ }\NormalTok{a}

    \CommentTok{# Non-homogenous poisson process}
\NormalTok{    evs <-}\StringTok{ }\KeywordTok{length}\NormalTok{(intervals) }\CommentTok{# lenght of arrival times }
\NormalTok{    nh_val <-}\StringTok{ }\DecValTok{0} \OperatorTok{+}\StringTok{ }\NormalTok{start }\CommentTok{# start of the event count}
\NormalTok{    nh_intervals <-}\StringTok{ }\KeywordTok{c}\NormalTok{() }\CommentTok{# arrival times for the NHPP}
    \ControlFlowTok{for}\NormalTok{ (i }\ControlFlowTok{in} \DecValTok{1}\OperatorTok{:}\NormalTok{evs) \{}
        \ControlFlowTok{if}\NormalTok{ (}\KeywordTok{runif}\NormalTok{(}\DecValTok{1}\NormalTok{) }\OperatorTok{<}\StringTok{ }\KeywordTok{fun}\NormalTok{(intervals[i])}\OperatorTok{/}\NormalTok{l) \{ }
            \CommentTok{# only including intervals from the HPP which}
            \CommentTok{# match with fun(intervals[i])/l probability}
\NormalTok{            nh_intervals <-}\StringTok{ }\KeywordTok{c}\NormalTok{(nh_intervals, intervals[i]) }
\NormalTok{            nh_val <-}\StringTok{ }\NormalTok{nh_val}\OperatorTok{+}\DecValTok{1} \CommentTok{# adding one to the event count}
\NormalTok{        \}}
\NormalTok{    \}}
\NormalTok{    nh_events <-}\StringTok{ }\KeywordTok{seq}\NormalTok{(}\DecValTok{1}\OperatorTok{+}\NormalTok{start,nh_val,}\DecValTok{1}\NormalTok{) }\CommentTok{# events since the previous group}
    \KeywordTok{return}\NormalTok{(}\KeywordTok{list}\NormalTok{(}\DataTypeTok{arrival_times=}\NormalTok{nh_intervals, }\DataTypeTok{events=}\NormalTok{nh_events))}
\NormalTok{\}}
\end{Highlighting}
\end{Shaded}

\newpage

\begin{Shaded}
\begin{Highlighting}[]
\NormalTok{simulation <-}\StringTok{ }\ControlFlowTok{function}\NormalTok{(iters, functions, lambdas, ints) \{}
    \CommentTok{# This function simulates from the NHPP}
    \CommentTok{# iters:     number of iterations to plot and add to the list of}
    \CommentTok{#            dataframes}
    \CommentTok{# functions: list of functions corresponding to the lambda function}
    \CommentTok{# lambdas:   list of lambdas for each subinterval}
    \CommentTok{# ints:      lists of vectors of 2 elements each containing the intervals}
    \CommentTok{#            that correspond to each element of lambdas and functions lists}
\NormalTok{    p <-}\StringTok{ }\KeywordTok{list}\NormalTok{()}
    \ControlFlowTok{for}\NormalTok{ (i }\ControlFlowTok{in} \DecValTok{1}\OperatorTok{:}\NormalTok{iters) \{}
\NormalTok{        maximum <-}\StringTok{ }\DecValTok{0} \CommentTok{# start for the next NHPP simulation to continue count}
\NormalTok{        arr_times <-}\StringTok{ }\KeywordTok{c}\NormalTok{() }\CommentTok{# arrival times}
\NormalTok{        events <-}\StringTok{ }\KeywordTok{c}\NormalTok{() }\CommentTok{# event counts}
        \ControlFlowTok{for}\NormalTok{ (k }\ControlFlowTok{in} \DecValTok{1}\OperatorTok{:}\DecValTok{4}\NormalTok{) \{}
\NormalTok{            int <-}\StringTok{ }\KeywordTok{non_hom_poisson}\NormalTok{(lambda_funs[[k]],lambdas[[k]],}
\NormalTok{                                   ints[[k]][}\DecValTok{1}\NormalTok{],ints[[k]][}\DecValTok{2}\NormalTok{],}
                                   \DataTypeTok{start=}\NormalTok{maximum)}
\NormalTok{            maximum <-}\StringTok{ }\KeywordTok{max}\NormalTok{(int}\OperatorTok{$}\NormalTok{events) }\CommentTok{# remembering last event count}
\NormalTok{            arr_times <-}\StringTok{ }\KeywordTok{c}\NormalTok{(arr_times, int}\OperatorTok{$}\NormalTok{arrival_times) }
\NormalTok{            events <-}\StringTok{ }\KeywordTok{c}\NormalTok{(events, int}\OperatorTok{$}\NormalTok{events)}
\NormalTok{        \}}
\NormalTok{        p[[i]] <-}\StringTok{ }\KeywordTok{data.frame}\NormalTok{(}\DataTypeTok{arrival_times=}\NormalTok{arr_times, }\DataTypeTok{events=}\NormalTok{events)}
        \CommentTok{# plots}
        \ControlFlowTok{if}\NormalTok{ (i }\OperatorTok{==}\StringTok{ }\DecValTok{1}\NormalTok{) \{}\KeywordTok{plot}\NormalTok{(arr_times, events, }\DataTypeTok{cex=}\FloatTok{0.5}\NormalTok{, }\DataTypeTok{pch=}\StringTok{'.'}\NormalTok{, }
                          \DataTypeTok{col=}\KeywordTok{randomColor}\NormalTok{(), }\DataTypeTok{xlim=}\KeywordTok{c}\NormalTok{(}\DecValTok{0}\NormalTok{,}\FloatTok{1.5}\NormalTok{), }
                          \DataTypeTok{ylim=}\KeywordTok{c}\NormalTok{(}\DecValTok{0}\NormalTok{,}\DecValTok{150}\NormalTok{))\}}
        \KeywordTok{lines}\NormalTok{(arr_times, events, }\DataTypeTok{col=}\KeywordTok{randomColor}\NormalTok{())}
\NormalTok{    \}}
    \KeywordTok{return}\NormalTok{(p)}
\NormalTok{\}}
\end{Highlighting}
\end{Shaded}

\newpage

\begin{Shaded}
\begin{Highlighting}[]
\NormalTok{data <-}\StringTok{ }\KeywordTok{simulation}\NormalTok{(}\DecValTok{1000}\NormalTok{, lambda_funs, lambdas, ints)}
\end{Highlighting}
\end{Shaded}

\includegraphics{./figures/unnamed-chunk-9-1.pdf}

\begin{Shaded}
\begin{Highlighting}[]
\NormalTok{ratio <-}\StringTok{ }\DecValTok{0}
\ControlFlowTok{for}\NormalTok{ (i }\ControlFlowTok{in} \DecValTok{1}\OperatorTok{:}\KeywordTok{length}\NormalTok{(data)) \{}
\NormalTok{    df <-}\StringTok{ }\KeywordTok{data.frame}\NormalTok{(data[[i]])}
\NormalTok{    cnt <-}\StringTok{ }\NormalTok{df }\OperatorTok\StringTok{ }\KeywordTok{filter}\NormalTok{(arrival_times }\OperatorTok{<}\StringTok{ }\FloatTok{0.91232} \OperatorTok{&}\StringTok{ }\NormalTok{events }\OperatorTok{>=}\StringTok{ }\DecValTok{150}\NormalTok{) }\OperatorTok\StringTok{ }\NormalTok{dplyr}\OperatorTok{::}\KeywordTok{count}\NormalTok{()}
    \ControlFlowTok{if}\NormalTok{ (cnt[}\DecValTok{1}\NormalTok{] }\OperatorTok{>=}\StringTok{ }\DecValTok{1}\NormalTok{) \{}
\NormalTok{        ratio <-}\StringTok{ }\NormalTok{ratio }\OperatorTok{+}\StringTok{ }\DecValTok{1}
\NormalTok{    \}}
\NormalTok{\}}
\NormalTok{ratio}\OperatorTok{/}\DecValTok{1000}
\CommentTok{#> [1] 0.192}
\end{Highlighting}
\end{Shaded}

\hypertarget{problem-3}{%
\section{Problem 3}\label{problem-3}}

\hypertarget{a-1}{%
\section{(a)}\label{a-1}}

Our infinitesimal generator is the following:

\(Q = \begin{pmatrix} - \lambda & \lambda & 0 & 0 & \dots \\ \mu & - (\lambda + \mu) & \lambda & 0 & \dots \\ 0 & 2 \mu & - (2 \mu + \lambda) & \lambda & \dots \\ \vdots & \vdots & \vdots & \vdots & \ddots \end{pmatrix}\)

\hypertarget{b-1}{%
\section{(b)}\label{b-1}}

We solve the following system:

\(\begin{cases} \sum_{i = 1}^{\infty} \pi_i = 1 \\ \lambda \pi_0 = \mu \pi_1 \\ \lambda \pi_1 = 2 \mu \pi_2 \\ \vdots \\ \lambda \pi_{n-1} = 2 \mu \pi_n \\ \vdots \\ \end{cases}\)

First we have:

\(\pi_1 = \frac{\lambda \pi_0}{\mu} \\ \pi_2 = \frac{\lambda^2 \pi_0}{2 \mu^2} \\ \pi_3 = \frac{\lambda^3 \pi_0}{2^2 \mu^3} \\ \vdots \\ \pi_n = \frac{\lambda^n \pi_0}{2^{n-1} \mu^n} \\ \vdots\)

Then:

\(\sum_{i=0}^{\infty} \pi_i = \pi_0 + \frac{\lambda \pi_0}{\mu} + \frac{\lambda^2 \pi_0}{2 \mu^2} + \dots + \frac{\lambda^n \pi_0}{2^{n-1} \mu^n} + \dots = 1\)

And so factoring \(\pi_0\) we get:

\(\pi_0 (1 + \frac{\lambda}{\mu} + \frac{\lambda^2}{2 \mu^2} + \dots) = \pi_0 (1 + \sum_{i=1}^{\infty} \frac{1}{2^{i-1}} (\frac{\lambda}{\mu})^{i})\)

Then multiplying \(\frac{2}{2}\) to the summation:

\(\pi_0 (1 + \frac{2}{2} \sum_{i=1}^{\infty} \frac{1}{2^{i-1}} (\frac{\lambda}{\mu})^{i})\)

= \(\pi_0 (2 \sum_{i=0}^{\infty} (\frac{\lambda}{2 \mu})^{i} - 1)\)

\(\pi_0 (2 (\frac{1}{1 - \frac{\lambda}{2 \mu}}) - 1) = 1\)

\(\pi_0 = \frac{1}{2 (\frac{1}{1 - \frac{\lambda}{2 \mu}}) - 1}\)

\(\vdots\)

\(\pi_n = \frac{\lambda^n}{2^{n-1}{\mu_n} \pi_0} \frac{1}{2 (\frac{1}{1 - \frac{\lambda}{2 \mu}}) - 1}\)

finally:

\(\pi_n = \frac{1}{2^{n-1}} (\frac{\lambda}{\mu})^{n} \pi_0\)

The infinite sum converges when \(|\frac{\lambda}{2 \mu}| < 1\) in which
case the stationary distribution P exists.

Then:

\(L = \sum_{n=0}^{\infty} \pi_n * n\)

Using the following sum:

\(\sum_{i=0}^{n-1} i a^i = \frac{a - na^n + (n-1)a^{n+1}}{(1-a)^2}\)

As n approaches infinity:

\(\sum_{i=0}^{\infty} i a^i = \frac{a}{(1-a)^2}\)

We get the following:

\(L = \pi_0 [1 + \sum_{n=0}^{\infty} (\frac{\lambda}{\mu})^n * \frac{n}{2^{n-1}}]\)

\(L = \pi_0 [1 + 2 \sum_{n=0}^{\infty} (\frac{\lambda}{2 \mu})^n * n]\)

\(L = \pi_0 [1 + 2 \frac{\frac{\lambda}{2 \mu}}{(1 - \frac{\lambda}{2 \mu})^2}]\)

\hypertarget{c-1}{%
\section{(c)}\label{c-1}}

Let's consider the probabilities conditioned on the number of customers
in the system that are present once our specific subject \emph{l} gets
into the system.

If there are no other customers when \emph{l} gets into the system,
there is no chance of overtaking.

\(P(N^{OV} = 0 | N^{PR} = 0) = 1\)

With \(N^{OV}\) being the number of customers that \emph{l} overtakes
and \(N^{PR}\) the number of customers present in the system (queing)
when \emph{l} gets in the system.

If \(N^{PR} = 1\), then \emph{l} can overtake only 1 customer, if the
time it takes to be served is shorter than the time it takes the other
customers to be served. Because of the memoryless property we can assert
the following:

\(P(N^{OV} = 0 | N^{PR} = 1) = \frac{\mu}{\mu + \mu} = \frac{1}{2}\)

Actually, in general:

\(P(N^{OV} = k | N^{PR} = n) = \frac{1}{n+1}\), \(n \leq c - 1\),
\(x = 0,1\)

As in this case c=2, our \emph{l} subject can' t overtake more than one
customer.

Now, if \(n \geq c\), that is, \emph{l} has to get in queue and wait to
be served. When \emph{l} gets served, there is also one more customer
getting served. Because, again, of the memoryless property.

\(P(N^{OV} = k | N^{PR} = n) = \frac{1}{c}\), \(n = c\), \(k = 0,1\)

In our case, it does not matter how many customers are in the system,
the probability of overtaking, conditioned to the number of customers
already in the system, is \(\frac{1}{2}\).

Now, using Bayes' theorem and the total probability rule, we can find
the probability of \emph{l} overtaking another customer.

\(P(A | B) = \frac{P(A \cap B)}{P(B)}\)

\(\frac{1}{2} \sum_{i=1}^{\infty} \pi_i = \frac{1}{2} \sum_{i=1}^{\infty} (\frac{1}{2^{i-1}}) (\frac{\lambda}{\mu})^{i} \pi_0\)

=
\(\sum_{i=1}^{\infty} (\frac{\lambda}{2 \mu})^i \pi_0 = (\sum_{i=0}^{\infty} ((\frac{\lambda}{2 \mu})^i) - 1) \pi_0\)

=
\((\frac{1}{1-\frac{\lambda}{2 \mu}} - 1) \pi_0 = \frac{1}{2 (\frac{1}{1 - \frac{\lambda}{2 \mu} - 1})} (\frac{1}{1 - \frac{\lambda}{\mu}} - 1)\)

=
\(\frac{1}{1 - \frac{\lambda}{2 \mu}} - 1 = \frac{1 - (1 - \frac{\lambda}{2 \mu})}{1 - \frac{\lambda}{2 \mu}} = \frac{\frac{\lambda}{2 \mu}}{1 - \frac{\lambda}{2 \mu}}\)

=
\(2 (\frac{1}{1 - \frac{\lambda}{2 \mu}}) - 1 = \frac{2}{1 - \frac{\lambda}{2 \mu}} - 1 = \frac{2 - (1 - \frac{\lambda}{2 \mu})}{1 - \frac{\lambda}{2 \mu}}\)

= \(\frac{1 + \frac{\lambda}{2 \mu}}{1 - \frac{\lambda}{2 \mu}}\)

Then:

\(\frac{\frac{\lambda}{2 \mu}}{1 \frac{\lambda}{2 \mu}} = \frac{\frac{\lambda}{2 \mu}}{\frac{2 \mu + \lambda}{2 \mu}} = \frac{\lambda}{2 \mu + \lambda}\)

So then we get:

\(P(N^{OV} = k) = \frac{\lambda}{2 \mu + \lambda}\), \(k = c-1 = 1\)

\hypertarget{d}{%
\section{(d)}\label{d}}

We define the following function to simulate the queueing system:

\begin{Shaded}
\begin{Highlighting}[]
\CommentTok{# Simulation of the System (M/M/2)}
\NormalTok{q <-}\StringTok{ }\ControlFlowTok{function}\NormalTok{(customers, l, m) \{}
  \CommentTok{# This function generates a M/M/2 queue system and returns a matrix }
  \CommentTok{# with columns of ArrivalTimes, exit=ExitTimes and service=ServiceTimes}
  \CommentTok{# PARAMS:}
  \CommentTok{# customers:   number of customers in the supermarket}
  \CommentTok{# l:     lambda for the homogenous poisson process }
  \CommentTok{#    (customers arrive to the unique cashiers waiting}
  \CommentTok{#    line according this rate)}
  \CommentTok{# m:     mu for the exponential distribution}
  \CommentTok{#    (The times to be served are independent and }
  \CommentTok{#    distributed as exponential with rate mu)}

  \CommentTok{# interval for arrival times}
\NormalTok{  exp_at <-}\StringTok{ }\KeywordTok{rexp}\NormalTok{(customers,l)}
  
  \CommentTok{# we have as many arrival times as the number of customers}
\NormalTok{  ArrivalTimes <-}\StringTok{ }\KeywordTok{rep}\NormalTok{(}\DecValTok{0}\NormalTok{,customers) }

\NormalTok{  ArrivalTimes[}\DecValTok{1}\NormalTok{] <-}\StringTok{ }\NormalTok{exp_at[}\DecValTok{1}\NormalTok{]}
  \ControlFlowTok{for}\NormalTok{ (i }\ControlFlowTok{in} \DecValTok{2}\OperatorTok{:}\NormalTok{customers) \{}
    \CommentTok{# arrival time = the previous arrival time + the interval between 2 customers}
\NormalTok{    ArrivalTimes[i] <-}\StringTok{ }\NormalTok{ArrivalTimes[i}\DecValTok{-1}\NormalTok{] }\OperatorTok{+}\StringTok{ }\NormalTok{exp_at[i]}
\NormalTok{  \} }
  
  \CommentTok{# service time distributed according to mu}
\NormalTok{  ServiceTimes <-}\StringTok{ }\KeywordTok{rexp}\NormalTok{(customers, m)}
  
  \CommentTok{# we have as many exit times as the number of customers}
\NormalTok{  ExitTimes <-}\StringTok{ }\KeywordTok{rep}\NormalTok{(}\DecValTok{0}\NormalTok{,customers)}
  
  \CommentTok{# the first two exit time is equal to the service time due to we have 2 cashiers}
\NormalTok{  ExitTimes[}\DecValTok{1}\OperatorTok{:}\DecValTok{2}\NormalTok{] <-}\StringTok{ }\NormalTok{ServiceTimes[}\DecValTok{1}\OperatorTok{:}\DecValTok{2}\NormalTok{]}
  \ControlFlowTok{for}\NormalTok{ (i }\ControlFlowTok{in} \DecValTok{3}\OperatorTok{:}\NormalTok{customers) \{}
    \CommentTok{# we sort exit time from larger to smaller}
\NormalTok{    SortedTimes <-}\StringTok{ }\KeywordTok{sort}\NormalTok{(ExitTimes[}\DecValTok{1}\OperatorTok{:}\NormalTok{(i}\DecValTok{-1}\NormalTok{)], }\DataTypeTok{decreasing=}\NormalTok{T)}
    
    \CommentTok{# all of the two cashiers are occupied, then the new customer will have to }
    \CommentTok{# wait until at least one of them leaves the supermarket (the faster one)}
    \CommentTok{# then the exit time of the new customer is exited time of the previous faster customer plus }
    \CommentTok{# the service time of this new customer}
    \ControlFlowTok{if}\NormalTok{ (ArrivalTimes[i] }\OperatorTok{<}\StringTok{ }\NormalTok{SortedTimes[}\DecValTok{2}\NormalTok{]) \{ExitTimes[i] <-}\StringTok{ }\NormalTok{SortedTimes[}\DecValTok{2}\NormalTok{] }\OperatorTok{+}\StringTok{ }\NormalTok{ServiceTimes[i]\}}
    
    \CommentTok{# one or two cashiers are free, then the exit time is the arrival time plus the service time}
    \ControlFlowTok{else}\NormalTok{ \{ExitTimes[i] <-}\StringTok{ }\NormalTok{ArrivalTimes[i] }\OperatorTok{+}\StringTok{ }\NormalTok{ServiceTimes[i]\}}
\NormalTok{  \}}
  
  \CommentTok{# create a matrix and return it}
\NormalTok{  times <-}\StringTok{ }\KeywordTok{data.frame}\NormalTok{(}\DataTypeTok{arrival=}\NormalTok{ArrivalTimes,}
                      \DataTypeTok{exit=}\NormalTok{ExitTimes,}
                      \DataTypeTok{service=}\NormalTok{ServiceTimes)}
  \KeywordTok{return}\NormalTok{(times)}
\NormalTok{\}}
\end{Highlighting}
\end{Shaded}

\begin{Shaded}
\begin{Highlighting}[]
\NormalTok{number_of_customers <-}\StringTok{ }\DecValTok{8500}
\NormalTok{times <-}\StringTok{ }\KeywordTok{q}\NormalTok{(number_of_customers, }\DataTypeTok{l=}\FloatTok{0.4}\NormalTok{, }\DataTypeTok{m=}\FloatTok{0.25}\NormalTok{) }
\KeywordTok{plot}\NormalTok{(times}\OperatorTok{$}\NormalTok{exit)}
\end{Highlighting}
\end{Shaded}

\includegraphics{./figures/unnamed-chunk-12-1.pdf}

\hypertarget{e}{%
\section{(e)}\label{e}}

We define the following function to calculate the overtaking
probability:

\begin{Shaded}
\begin{Highlighting}[]
\NormalTok{d <-}\StringTok{ }\ControlFlowTok{function}\NormalTok{(customers, queue, ExitTimes) \{}
  \CommentTok{# This function generates a M/M/2 queue system and returns }
  \CommentTok{# the probability of overtaking}
  \CommentTok{# PARAMS:}
  \CommentTok{# customers:   number of customers in the supermarket}
  \CommentTok{# queue:       number of people in the quene}
  \CommentTok{# ExitTimes:   the exited times of each customer}

  \CommentTok{# we define the number of people of overtaking}
\NormalTok{  ot <-}\StringTok{ }\DecValTok{0}
  \ControlFlowTok{for}\NormalTok{ (i }\ControlFlowTok{in}\NormalTok{ queue}\OperatorTok{:}\NormalTok{(customers }\OperatorTok{-}\StringTok{ }\NormalTok{queue)) \{}

    \CommentTok{# the number of people overtaking of all customers is the previous }
    \CommentTok{# number plus 1 if the exit time of the i-th customer is smaller (earlier) }
    \CommentTok{# than the (i-1)th customer }
\NormalTok{    ot <-}\StringTok{ }\NormalTok{ot }\OperatorTok{+}\StringTok{ }\KeywordTok{sum}\NormalTok{(ExitTimes[i] }\OperatorTok{<}\StringTok{ }\NormalTok{ExitTimes[}\DecValTok{1}\OperatorTok{:}\NormalTok{(i}\DecValTok{-1}\NormalTok{)])}
\NormalTok{  \}}
\NormalTok{  r <-}\StringTok{ }\NormalTok{customers}\DecValTok{-2}\OperatorTok{*}\NormalTok{queue}
\NormalTok{  ot_prob <-}\StringTok{ }\NormalTok{ot}\OperatorTok{/}\NormalTok{r}
  \KeywordTok{return}\NormalTok{(ot_prob)}
\NormalTok{\}}
\end{Highlighting}
\end{Shaded}

Write theRcode necessary to simulate the system (provide the code) and
generate thetimes customers leave the supermarket

\begin{Shaded}
\begin{Highlighting}[]
\NormalTok{avgs <-}\StringTok{ }\KeywordTok{c}\NormalTok{() }\CommentTok{# general average of multiple simulations}
\NormalTok{number_of_customers <-}\StringTok{ }\DecValTok{8500} \CommentTok{# simulations done with 8500 people}
\ControlFlowTok{for}\NormalTok{ (i }\ControlFlowTok{in} \DecValTok{1}\OperatorTok{:}\DecValTok{10}\NormalTok{) \{}
\NormalTok{    times <-}\StringTok{ }\KeywordTok{q}\NormalTok{(number_of_customers, }\DataTypeTok{l=}\FloatTok{0.4}\NormalTok{, }\DataTypeTok{m=}\FloatTok{0.25}\NormalTok{) }\CommentTok{# running the simulation}
\NormalTok{    avg <-}\StringTok{ }\KeywordTok{c}\NormalTok{() }\CommentTok{# estimation (usually includes an extra person)}
\NormalTok{    avg_p <-}\StringTok{ }\KeywordTok{c}\NormalTok{() }\CommentTok{# pessimistic estimation (usually excludes that extra person)}
    \ControlFlowTok{for}\NormalTok{ (i }\ControlFlowTok{in} \DecValTok{0}\OperatorTok{:}\DecValTok{99}\NormalTok{) \{}
        \CommentTok{# assuming the person number number_of_customers - i is the last one}
\NormalTok{        last_arrival_time <-}\StringTok{ }\NormalTok{times}\OperatorTok{$}\NormalTok{arrival[}\KeywordTok{length}\NormalTok{(times}\OperatorTok{$}\NormalTok{arrival)}\OperatorTok{-}\NormalTok{i] }
        \CommentTok{# checking all exit times mayores al arrival time of the person number number_of_customers - i}
\NormalTok{        val <-}\StringTok{ }\KeywordTok{sum}\NormalTok{(times}\OperatorTok{$}\NormalTok{exit[}\DecValTok{1}\OperatorTok{:}\NormalTok{(}\KeywordTok{length}\NormalTok{(times}\OperatorTok{$}\NormalTok{arrival)}\OperatorTok{-}\NormalTok{i}\DecValTok{-1}\NormalTok{)] }\OperatorTok{>}\StringTok{ }\NormalTok{last_arrival_time) }
\NormalTok{        avg <-}\StringTok{ }\KeywordTok{c}\NormalTok{(avg,val)}
\NormalTok{        avg_p <-}\StringTok{ }\KeywordTok{c}\NormalTok{(avg,val}\DecValTok{-1}\NormalTok{)}
\NormalTok{    \}}
    \CommentTok{# joining all estimations }
\NormalTok{    total_avg <-}\StringTok{ }\KeywordTok{mean}\NormalTok{(}\KeywordTok{c}\NormalTok{(avg,avg_p))}
    \CommentTok{# adding to vector with  all averages}
\NormalTok{    avgs <-}\StringTok{ }\KeywordTok{c}\NormalTok{(avgs, total_avg) }
\NormalTok{\}}
\CommentTok{# printing averages}
\NormalTok{avgs}
\CommentTok{#>  [1]  2.278607  4.228856  2.636816 11.771144  2.064677  3.621891  2.696517}
\CommentTok{#>  [8]  5.905473  2.179104 10.149254}
\CommentTok{# calculating the average of averages}
\KeywordTok{mean}\NormalTok{(avgs) }
\CommentTok{#> [1] 4.753234}
\end{Highlighting}
\end{Shaded}

Calculation by hand:

\begin{Shaded}
\begin{Highlighting}[]
\NormalTok{lambda <-}\StringTok{ }\FloatTok{0.4}
\NormalTok{mu <-}\StringTok{ }\FloatTok{0.25}
\NormalTok{lambda2mu <-}\StringTok{ }\NormalTok{lambda}\OperatorTok{/}\NormalTok{(}\DecValTok{2}\OperatorTok{*}\NormalTok{mu)}

\NormalTok{pi_}\DecValTok{0}\NormalTok{ <-}\StringTok{ }\DecValTok{1}\OperatorTok{/}\NormalTok{(}\DecValTok{2}\OperatorTok{*}\NormalTok{(}\DecValTok{1}\OperatorTok{/}\NormalTok{(}\DecValTok{1}\OperatorTok{-}\NormalTok{lambda2mu)) }\OperatorTok{-}\StringTok{ }\DecValTok{1}\NormalTok{)}
\NormalTok{L <-}\StringTok{ }\NormalTok{pi_}\DecValTok{0} \OperatorTok{*}\StringTok{ }\NormalTok{(}\DecValTok{1} \OperatorTok{+}\StringTok{ }\DecValTok{2}\OperatorTok{*}\NormalTok{(lambda2mu}\OperatorTok{/}\NormalTok{((}\DecValTok{1}\OperatorTok{-}\NormalTok{lambda2mu)}\OperatorTok{^}\DecValTok{2}\NormalTok{)))}
\end{Highlighting}
\end{Shaded}

\(\frac{\lambda}{2 \mu} = \frac{0.4}{2(0.25)} = 0.8\)

\(\pi_0 = \frac{1}{2(\frac{1}{1-0.8}) - 1} = \frac{1}{9} = \approx 0. \bar{1}\)

\(L = \pi_0 (1 + 2 \frac{0.8}{(1-0.8)^2}) = 4. \bar{5}\)

\end{document}
